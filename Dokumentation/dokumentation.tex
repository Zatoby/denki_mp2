\documentclass[12pt,a4paper]{article}
\usepackage[utf8]{inputenc}
\usepackage[austrian]{babel}
\usepackage[T1]{fontenc}
\usepackage{amsmath}
\usepackage{amsfonts}
\usepackage{amssymb}
\usepackage[hidelinks]{hyperref}
\usepackage[left=2cm,right=2cm,top=2cm,bottom=2cm]{geometry}
\author{Tobias Hernandez Perez (12526308) und Laurenz Reinthaler (12519008)}
\title{MP2 - Denki}

\begin{document}
\urldef{\cairobook}\url{https://books.google.at/books?id=qP2KDwAAQBAJ&pg=PT84&hl=de&source=gbs_selected_pages&cad=1#v=onepage&q&f=false}

\maketitle

\section*{Ideenfindung und Recherche}
Bei unserem Ideen-Brainstorming ist eine Idee für das Artefakt besonders herausgestochen, da es vor allem die beiden Themengebiete Scientific und Design Thinking elegant kombiniert. Und zwar geht es um die Thematik von verzerrt dargestellten Statistiken bzw. Graphen. Um genau zu sein, wollten wir eine Webseite bauen, wo man selber mit den Werten eines Graphen hantieren kann. Damit wollen wir Bewusstsein schaffen, wie Graphen fälschlich dargestellt werden können, um z.B. Meinungen zu manipulieren.

Während unserer Recherche sind wir über einige interessante Quellen gestolpert, zum Beispiel ein Buch von Alberto Cairo\footnote{\cairobook}, welches unsere exakte Thematik beschreibt. Oder auch ein Webtool zur Extraktion von den Werten eines Graphen\footnote{\url{https://www.graphreader.com/}}

Um auf der Website Graphen darstellen zu können, haben wir uns für die Graphen-Bibliothek chart.js\footnote{https://www.chartjs.org/} entschieden.


Die Recherche hat uns in unserer Idee nur bestärkt, und uns vor allem Inspiration für das Layout und den Aufbau der Website gegeben. In dem Sinne haben wir uns gleich an die Umsetzung gemacht.

\section*{Umsetzung}
Die chart.js Bibliothek hat das grobe Gerüst unseres Projekts dargestellt. Wir mussten rundherum etwas erschaffen, dass eine benutzerfreundliche Interaktion mit den Graphen ermöglicht.
Um das effizient umsetzen zu können, haben wir \textbf{Git} in Verbindung mit \textbf{GitHub} genutzt, um simultan an dem Code arbeiten zu können.

Die Kernfunktionen waren schnell erstellt. Damit die Seite ansprechend ist, musste sie allerdings auch gestylt werden. Dazu haben wir \textbf{Tailwindcss} genutzt.
Tailwind ist ein CSS Framework, das das stylen von Websites durch vordefinierte Klassen weitaus einfacher macht.


\section*{Feedback}
\subsection*{BOKU Studentin}
\textbf{Wofür könnte man das Tool einsetzen?}
Um Graphen so zu zeigen/manipullieren, dass sie die eigene Meinung vertreten.
\\\\
\textbf{Hast du dich zurechtgefunden?}
Ja, mehr oder weniger, nach etwas ausprobieren.
\\\\
\textbf{Ist das Design ansprechend?/passend}
Ja.
\\\\
\textbf{Welche Features fehlen dir?}
Keine, die mir spontan einfallen.
\\\\
\textbf{Willst du uns sonst noch etwas mitteilen?}
Mehrere Farben würden vielleicht helfen und es etwas übersichtlicher machen.


\subsection*{Web Designerin}
\textbf{Wofür könnte man das Tool einsetzen?}
Darstellung von statistischen Grafiken.
\\\\
\textbf{Hast du dich zurechtgefunden?}
Ja.
\\\\
\textbf{Ist das Design ansprechend?/passend}
Ja, es ist ansprechender als R.
\\\\
\textbf{Welche Features fehlen dir?}
Export & Import.
\\\\
\textbf{Willst du uns sonst noch etwas mitteilen?}
Farb-(Kontrast-)Modus auswählbar.

\subsection*{Limitationen des Feedbacks}
Wir haben nur Personen in unserem näheren Umfeld befragt. Außerdem war es keine anonymisierte Umfrage, was die Antworten beeinflusst haben könnte. Die Altersgruppen weisen auch viele Lücken auf,
bspw. wurden weder Kinder noch Senioren befragt.

\end{document}
