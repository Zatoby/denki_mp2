\documentclass[12pt,a4paper]{article}
\usepackage[utf8]{inputenc}
\usepackage[austrian]{babel}
\usepackage[T1]{fontenc}
\usepackage{amsmath}
\usepackage{amsfonts}
\usepackage{amssymb}
\usepackage[hidelinks]{hyperref}
\usepackage[left=2cm,right=2cm,top=2cm,bottom=2cm]{geometry}
\author{Tobias Hernandez Perez (12526308) und Laurenz Reinthaler (12519008)}
\title{MP2 - Denki}

\begin{document}
\urldef{\cairobook}\url{https://books.google.at/books?id=qP2KDwAAQBAJ&pg=PT84&hl=de&source=gbs_selected_pages&cad=1#v=onepage&q&f=false}

\maketitle

\section*{Ideenfindung und Recherche}
Bei unserem Ideen-Brainstorming ist eine Idee für das Artefakt besonders herausgestochen, da es vor allem die beiden Themengebiete Scientific und Design Thinking elegant kombiniert. Und zwar geht es um die Thematik von verzerrt dargestellten Statistiken bzw. Graphen. Um genau zu sein, wollten wir eine Webseite bauen, wo man selber mit den Werten eines Graphen hantieren kann. Damit wollen wir Bewusstsein schaffen, wie Graphen fälschlich dargestellt werden können, um z.B. Meinungen zu manipulieren.

Während unserer Recherche sind wir über einige interessante Quellen gestolpert, zum Beispiel ein Buch von Alberto Cairo\footnote{\cairobook}, welches unsere exakte Thematik beschreibt. Oder auch ein Webtool zur Extraktion von den Werten eines Graphen\footnote{\url{https://www.graphreader.com/}}

Um auf der Website Graphen darstellen zu können, haben wir uns für die Graphen-Bibliothek chart.js\footnote{https://www.chartjs.org/} entschieden.


Die Recherche hat uns in unserer Idee nur bestärkt, und uns vor allem Inspiration für das Layout und den Aufbau der Website gegeben. In dem Sinne haben wir uns gleich an die Umsetzung gemacht.

\section*{Feedback}
TODO

\end{document}
